
\chapter{Resulterande läromaterial}

\begin{draft}

Läromaterialet blev i slutändan en sammanvävning av domänspecifika språk som
modellerar fysik, och en lärotext som förklarar kopplingen mellan fysiken och de
domänspecifika språken. Figur~\ref{fig:smakprov_laromaterial} visar ett kort
utdrag ur läromaterialet. Där ser man hur domänspecifika språk och lärotext är
sammanvävda. Ett längre utdrag finns i bilaga \ref{cha:utdrag}.

\begin{figure}[tph]
  \includegraphics[width=\linewidth]{figure/smakprov_laromaterial.png}
  \caption{Ett smakprov över hur det resulterande läromaterialet ser ut.
           Lärotexten ligger mot den ljusgrå bakgrunden medan det
           domänspecifika språket ligger mot den
           mörkgrå.}~\label{fig:smakprov_laromaterial} 
\end{figure}

I läromaterialet finns även bilder och övningar. Figur~\ref{fig:smakprov_bild_laromaterial} är ett exempel på en bild ur läromaterialet. Notera speciellt den medvetet oseriösa ritningstekniken som är tänkt att vara rolig och muntra upp läsaren. Övningar ligger både i den löpande texten och i slutet av kapitlet. Övningarna i den löpande texten innebär oftast att läsaren ska implementera en liten del av det aktuella domänspecifika språket på egen hand. Det här illustreras i figur~\ref{fig:smakprov_ovning}.

\begin{figure}[tph]
    \centering
    \begin{subfigure}[t]{0.5\textwidth}
        \centering
        \includegraphics[width=0.9\linewidth]{figure/smakprov_bild_laromaterial.png}
        \caption{Exempel på en bild. Bilden visar hur en hund springer och
                 hoppar upp på en stillastående
                 vagn.}~\label{fig:smakprov_bild_laromaterial}
    \end{subfigure}% 
    ~~~
    \begin{subfigure}[t]{0.5\textwidth}
        \centering
        \includegraphics[width=0.9\linewidth]{figure/smakprov_ovning.png}
        \caption{Exempel på en övning. Övningen ligger som en del av den
                 löpande texten.}~\label{fig:smakprov_ovning}
    \end{subfigure}
    \caption{Exempel på en bild och en övning ur läromaterialet.} 
\end{figure}

Läromataterialet behandlar ett flertal områden inom fysik och matematik.
Fokuset är på klassisk mekanik samt till det området tillhörande matematik. I
sin fullständighet är de behandlade områdena:

\begin{itemize}
  \item Bevis
  \item Dimensioner
  \item Matematisk analys
  \item Vektorer
  \item Kompositioner och tillämpningar av ovanstående
\end{itemize}

I \textit{bevis}-kapitlet presenteras bevisföring med hjälp av Haskells typsystem. Närmare bestämt används \textit{Curry Howard isomorfin}, som säger att typer är påståenden och värden bevis.\cite{chi} Det exemplifieras genom att kinematiska formler bevisas. 

\textit{Dimensioner} behandlar dimensioner, storheter och enheter inom fysiken. Fysikaliska dimensioner införs på typnivå i Haskell för att visa likheten mellan Haskells typsystem och hur man måste förhålla sig till dimensioner inom fysiken. Typnivå-programmering\footnote{Vanligtvis manipuleras \textit{värden} när man programmerar, i Haskell och i andra spårk. Typnivå-programmering är precis som vanlig programmering med skillnaden att den sker på typnivån, det vill säga, typer modifieras. Läromaterialet \cite{LYAP} hänvisas till för en utförligare förklaring.} används för att göra likheterna så tydliga som möjligt.

I \textit{matematisk analys} konstrueras ett syntaxträd för algebraiska uttryck och funktioner. Därefter implementeras symbolisk derivering och integrering på syntaxträdet.

\textit{Vektorer} implementerar vektorer och vektoroperationer i Haskell. Lagar som ska gälla för vektorer implementerades och testades.

I läromataterialet finns, förutom de fyra ovanstående grundläggande områdena,
även tillämpningar av dem på exempelproblem. Till exempel används dimensioner
till att lösa problem med fritt fall. De tillämpningar som behandlats är

\begin{itemize}
  \item Gundbräda
  \item Krafter på lådor
\end{itemize}

Gör PSSSO\footnote{På Samma Sätt Som Ovan} för de kompisita områdena, dvs skriv vad varje handlar om.

Läromaterialet blev publicerat på en hemsida\cite{LYAP} och all källkod finns
tillgänglig på projektets GitHub-repository.\cite{LYAP_repo} Texten är skriven
på engelska.

\end{draft}

%\begin{binge}
%
%EXEMPEL PÅ ATT VI ANVÄNDER LITERAT STIL, OM DET SKULLE BEHÖVAS
%
%I Quantity så är detta tydligt. Då gjordes en `taste of types'' tidigt för att
%läsaren skulle förstå Quantity bättre, innan massa detaljer gicks %in på. Frågan
%är dock om detta ska gås in på i rapporten?
%
%\end{binge}































