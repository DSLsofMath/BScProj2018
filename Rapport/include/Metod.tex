% Metoddelen redogör för vad du gjort och hur du gått tillväga; det är
% en beskrivning av den metod som ligger till grund för det du kommit
% fram till och hävdar i din rapport.

% Beskrivningen i metoddelen ska vara koncis snarare än helt
% uttömmande men ska samtidigt göra det möjligt att upprepa studien

% Metoddelen ska inte vara en omgjord labbinstruktion och den ska inte
% heller innehålla teori med mindre än att teoretiska hänsyn har haft
% en direkt inverkan på metoden.

% Metoddelen skrivs nästan alltid i dåtid (imperfekt) och ofta används
% passiv form för att beskriva forskningsaktiviteter.

\chapter{Genomförande}

Projektets genomförande bestod av fyra delar. Den största delen var
konstruktionen av själva läromaterialet i vilken det ingick sökande efter lämpliga
områden, implementation av domänspecifika språk och skrivande av lärotext. De
tre andra delarna var publicering av läromaterialet på en hemsida, utvärdering
av läromaterialet med en testgrupp samt möten med Åke Fäldt, examinator och
föreläsare för Fysik för ingenjörer. Mötena med Fäldt hade två syften: att hitta
problemområden i fysikkursen och att få återkoppling på läromaterialet. Alla de olika delar i projektet genomfördes samtidigt men de finns här beskrivna separat.

\section{Konstruktion av läromaterialet}\label{sec:konstruktion}

Läromaterialet består av fem kapitel som vardera behandlar separata områden. Skapandet av dem skedde fristående men de innehöll alla de tre faserna sökande, implementation och skrivande, som såg likartade ut för dem alla. Det fanns dock visst överlapp mellan de fristående processerna. Sökandet gav ofta flera områden samtidigt och implementation och skrivande genomfördes ofta parallellt. För att tydliggöra processerna är de dock beskrivna separat.

\subsection{Sökande efter områden att behandla}\label{sec:valet}

Ett domänspecifikt språk modellerar ett specifikt och avgränsat område. Därför
var det naturligt att söka och tänka i termer av avgränsade områden inom
fysiken. För att rent praktiskt hitta områden att behandla kontaktades Åke
Fäldt. Dessutom studerades kursens bok (University
Physics~\cite{UP}) och dess övriga material.

Denna sökandeprocess innefattade inte bara att \textit{hitta} fysikaliska
områden utan även \textit{organisera} dem i relation till varandra. Som framgår
senare är vissa områden baserade på andra. Denna organisering
var viktig för att kunna implementera de domänspecifika språken på bästa sätt utan överlappningar och presentera dem i en pedagogisk ordning.

\subsubsection*{Kontakt med fysikläraren}
\label{sec:kontakt_faldt}

Fäldt tillfrågades om vilka områden han i allmänhet anser att studenter har
svårt för. Detta för att i enlighet med projektets mål börja med de, för
studenterna, problematiska områdena. Enligt Fäldt är ett allmänt problem att
egna mentala modeller för problem är felaktiga eftersom studenter ofta tar
genvägar som inte bygger på saker de är säkra stämmer. En annan erfarenhet
är att så länge den första raden i en uppgiftslösning är rätt, så brukar
resten också vara rätt. Med andra ord, har studenten väl identifierat vilken typ av
problem det rör sig om brukar det inte vara några svårigheter att lösa
uppgiften. Sist men inte minst ansåg Fäldt att det fattades grundläggande
kunskaper om matematisk analys hos studenterna.

Med hjälp av insikterna från Fäldt drogs två slutsatser. Den första slutsatsen
var att matematisk analys var ett område värt att behandla i detalj. Den andra
var att genom att ge struktur till olika typer av problem kan det
förhoppningsvis underlätta för studenter att lära sig identifiera vilken
typ av uppgift de handskas med.

\subsubsection*{Studerande av kursbok och kursmaterial}

Efter kontakten med Fäldt kunde ett sökande efter konkreta områden genomföras.
Detta gjordes genom att studera kursboken och kursmaterialet tillhörande Fysik
för ingenjörer. Innehållet som hittades delades upp i avgränsade områden för att
de skulle bli lämpade för varsitt domänspecifikt språk. Av speciellt intresse
var de kapitel som behandlade mekanik (i enlighet med projektets mål att börja
med klassisk mekanik), matematisk analys samt de kapitel som använde sig av en
specifik syntax. Domänspecifik syntax var av intresse att finna
då en betydlig del av domänspecifika språk är modellering av just syntaxen.

Sökandet i kursboken och kursmaterialet gav viktiga kunskaper om områden att
behandla. Men minst lika viktiga var de inledande experiment som gjordes på
varje område för att se huruvida det lämpade sig att göra ett domänspecifikt
språk av och hur det skulle kunna se ut. Experimenten visade att enbart vissa
områden, till exempel vektorer, fungerade bra att göra ett domänspecifikt språk
av. Andra områden, till exempel lutande plan, var mindre lämpliga. Vad som skiljer dem åt är att vektorer har tydliga data och operationer (till exempel skalärprodukt) medan lutande plan har egenskaper (till exemel friktionskoefficienter och vinklar) som är relaterade till varandra med ekvationer. Det här diskuteras utförligare i avsnitt~\ref{sec:lampligt}. Det framgick också att det
blev ett överlapp mellan olika domänspecifika språk trots att områdena var fristående.
Ett exempel var det domänspecifika språk för partikelmekanik som till stor del
liknade de domänspecifika språken för matematisk analys och vektorer.

Det blev av dessa skäl nödvändigt att göra en distinktion mellan två typer av
områden: \textit{grundläggande} och \textit{komposita}. Grundläggande områden är
helt fristående från andra områden och behandlar grundläggande koncept.
Komposita områden bygger vidare på andra områden eller tillämpar andra områden
på konkreta fysikaliska problem.

\subsubsection*{Områden som valdes ut}

När kunskap inhämtats om olika områden kunde ett urval göras. De områden som
identifierades som grundläggande och som hade en väl lämpad struktur enligt avsnittet innan valdes ut. Med detta som grund blev områdena som valdes ut fysikaliska dimensioner, matematisk analys och vektorer. Här följer en kortfattad motivering av valet av dem.

\textit{Dimensioner} eftersom det är viktigt för studenter att förstå
hur dimensioner påverkas av algebraiska operationer. Det kan också vara
hjälpsamt att utföra automatisk, datorassisterad dimensionsanalys på
beräkningar.

\textit{Matematisk analys} eftersom alla koncept inom klassisk mekanik är
relaterade genom matematisk analys. Mer specifikt används
differenser\footnote{Till exempel används $\Delta(x)$ för att beskriva
förflyttning i $x$-led.} för att beskriva medelrörelse, och infinitesimaler
för att beskriva momentanrörelser. Vidare var matematisk analys just det
område som Fäldt pekade ut som speciellt viktigt och något som studenter har
svårt för.

\textit{Vektorer} eftersom det är en viktig grundsten inom den klassiska
mekaniken. Alla krafter, hastigheter och accelerationer betraktas som vektorer i
planet eller rummet, och dessa är alla fundamentala element inom klassisk
mekanik.

De komposita områdena identifierades som områden som byggde vidare på de redan
implementerade grundläggande områdena. De komposita områdena som valdes ut
blev exempelproblem och partikelmekanik. Här följer en kortfattad motivering av valet av dem.

\textit{Exempelproblem} för att visa hur ett par typuppgifter i klassisk mekanik kan modelleras i något av läromaterialets domänspecifika språk. Närmare bestämt tillämpas de domänspecifika språken på \textit{krafter på lådor} och \textit{gungbräda}.

\textit{Partikelmekanik} för att visa hur de grundläggande områdena kan kombineras till ett domänspecifikt språk som är mer fysikorienterat än de tre grundläggande. Dessutom är partikelmekanik fundamental i klassisk mekanik.

\subsection{Implementation av domänspecifika språk för områdena}

Implementationen av domänspecifika språk var en iterativ process.
Den inleddes med att
bygga vidare på den experimentering som gjorts under urvalsfasen. Det finns inte bara
ett rätt sätt att skriva ett domänspecifikt språk på, därav gjordes försök med
flera olika varianter för att se vad som fungerade bäst.  I flera fall har implementationer gjorts om från grunden om
det visat sig att implementationen kunde gjorts bättre eller
hade brister. Dessutom gjordes, i varierande mån, fördjupande litteraturstudier av domänspecifika
språk, fysik och Haskell för att kunna implementera de domänspecifika språken på bästa sätt.

Vad som ansågs vara en bra, eller åtminstone tillräckligt bra implementation
var i huvudsak baserat på gruppmedlemmarnas erfarenhet av Haskell och diskussion
inom gruppen och med handledaren, det viktigaste var att de var
lättförståeliga. Därför användes inte alltid den programtekniskt elegantaste
implementationen utan den längre versionen föredrogs för att göra
läromaterialet så lättläst som möjligt. Dock avstods det inte från användning av
mer avancerade funktioner i Haskell när materialet motiverade dem, men då alltid
med en uttömmande förklaring om hur det fungerade och utan krav på tidigare
kunskap hos läsaren.

Efter att ett domänspecifikt språk implementerats skrevs tester till det. Det
som var intressant att testa var olika lagar som skulle gälla, och  eftersom de
domänspecifika språken i läromaterialet modellerade matematik var det matematiska lagar
som skulle gälla. Ett exempel var att vektoraddition skulle vara kommutativ.
Testerna gjordes med hjälp av \textit{QuickCheck}~\cite{QC} vilket är ett
testningsverktyg i Haskell som genererar många och slumpmässiga testfall. Att
lagarna gällde för de domänspecifika språken verifierades med andra ord genom
testa för ett stort antal exempelvärden. Inga bevis, utan enbart tester, gjordes för att kontrollera att lagarna gällde.

\subsection{Skriva lärotext}

I samband med att ett område implementerades skrevs också den tillhörande
lärotexten. Till en början skrevs lärotext som fokuserade på att förklara
programkoden. Detta var ett naturligt val eftersom det var viktigt att
programkoden gick att förstå innan kopplingar till matematik och fysik kunde
förklaras, lärotext av det slaget skrevs nämligen senare. Avslutningsvis skrevs
en inledning och avslutning till kapitlet.

Generellt under skrivningen togs det hänsyn till en specifik underaspekt i
ARCS-modellen, nämligen \textit{humor}. Språket i lärotexten har varit lättsamt,
vardagligt och talspråkligt för att hålla kvar uppmärksamheten hos läsaren. Det
har även ritats roliga bilder för att ge ytterligare humoristiska drag.

Lärotexten och programkoden skrevs sammanvävt i samma fil, i Literate
Haskell. Litterat programmering passade bra ihop med
hur läromaterialet skulle se ut då det betonade det jämnbördiga förhållandet
mellan programkod och förklaringar. För att läromaterialet skulle vara
lättförståeligt var det också viktigt att presentera materialet i den ordning
som en mänsklig läsare, och inte datorn, tyckte var enklast. Avsnitt \ref{sec:lhs} beskriver hur litterat programmering fungerar i allmänhet och ger
en bra bild av hur det såg ut även i detta projekt.

Under skrivandet av lärotexten lades övningar till. Dessa skapades genom att
modifiera befintlig lärotext, istället för att förklara allting uppmanar den
läsaren då och då att göra nästa steg i implementationen själv. När ett kapitel
var avslutat lades dessutom extra övningar till i slutet, dessa övningar var
ofta vidareutvecklingar av det domänspecifika språk som redan implementerats.

Skrivandet av lärotexten till de grundläggande och komposita områdena var
övergripande likadana. Skillnaden låg i balansen mellan Haskell och fysik. För
de grundläggande områdena fokuserade lärotexten mer på Haskell eftersom det var
ett helt nytt domänspecifikt språk som skulle konstrueras. Hur det fungerade var
därför viktigt att förklara. För de komposita områdena låg däremot ett större fokus på fysik. För dessa områden visades hur de
domänspecifika språken var praktiskt användbara och då förklarades fysiken, för
att sedan visa hur den fysiken kunde representeras i de domänspecifika
språken.

Ett exempel på ovanstående är kapitlet kring det komposita området partikelmekanik. Dess implementation var en sammanslagning av området vektorer och
matematisk analys där fokus flyttats till att visa hur det direkt gick att översätta de
fysikaliska formlerna som beskriver partiklars rörelse och energier till
Haskell-kod med hjälp av de grundläggande områdena. Beskrivning av relationen arbete-energi (engelska \textit{Work-Energy theorem}) gick då till som i figur \ref{fig:komposit-ex}.

\begin{figure}[tph]
  \centering
  \fbox{\includegraphics[width=1\textwidth]{figure/komposit-Ex.png}}
  \caption{Implementation av relationen arbete-energi i läromaterialet}
  \label{fig:komposit-ex}
\end{figure}

Implementationen kan kompileras och testas och ska då visa på att
implementationerna av de grundläggande områdena är både rigorösa och korrekta.
Det visar dessutom att det går att använda materialet som presenteras tidigare
till att implementera och lösa mer komplexa problem.

\section{Skapande av och publicering på hemsidan}

Läromaterialet kompilerades med hjälp av ett skript och
publicerades på en hemsida. Skriptet anropar
Pandoc för att konvertera från källkod i Literate
Haskell-format till HTML, redo att visas på en hemsida. Pandoc
paketerar även med \textit{MathJax} som använder JavaScript för att
rendera matematiska formler i LaTeX-format på fint och läsbart
vis. Utan stöd för JavaScript skrivs matematik ut som omodifierad
LaTeX-kod, vilket är mer svårläst, men fortfarande tolkningsbart. Det
skrevs även CSS-kod för att modifiera utseendet av
hemsidan för att den skulle bli prydligare och mer lättläst.

Varje källfil betraktades som ett kapitel och publicerades som en
separat undersida. Med hjälp av ett index beskrivet i skriptet
konstruerades navigationselement mellan kapitel på varje undersida
och en innehållsförteckning.

För publicering lades all data producerad av skriptet i en ny git-gren (engelska \textit{git branch}) med namnet \texttt{gh-pages}. Att alla grenar synkroniseras
mot GitHub medför att alla filer på \texttt{gh-pages} grenen
visas som en hemsida med hjälp av \textit{GitHub
  Pages}. Publiceringen skedde inte kontinuerligt eller automatiskt,
utan krävde en manuell synkronisering vid varje önskad uppdatering av
hemsidan.

\section{Utvärdering med testgrupp}

  % Återkoppling från examinator (NAD): "Nils Anders Danielsson <nad@cse.gu.se>
  % 27 Feb (1 day ago)
  % to Patrik, Andreas
  % Hi,Your BSc project groups both try to make tools for learning. I had some
  % discussion with Andreas' group about their plans for evaluating how well
  % their product works. My position is that, given the resource limits of
  % these projects (and general problems of reproducibility in social
  % sciences), it is very hard to perform an evaluation that gives useful
  % results. I don't mind if your groups try to perform some kind of
  % evaluation, but I suggest that you tell them to avoid overstating the
  % importance of the evaluations in the final reports."
  % Jag tror det är kompatibelt med det jag sagt tidigare - att göra en "ordentlig" utvärdering av det pedagogiska utfallet är komplicerat och tar (kalender-)tid.
  % Informell utvärdering av en testgrupp bör dock ingå.

För att utvärdera läromaterialet gjordes en kort och informell utvärdering med
en testgrupp. Testgruppen bestod av tre andra studenter på Chalmers som gick
tredje året på Datateknik och Informationsteknik. De hade alla läst Fysik för
ingenjörer eller motsvarande tidigare och även läst en kurs i Haskell.
Däremot hade de inte läst DSLsofMath eller motsvarande. Domänspecifika språk var
med andra ord nytt för dem.

Utvärderingen gjordes genom att visa dem läromaterialet med en kort
presentation och bakgrund. Sedan fick de på egen hand läsa materialet och deras
spontana reaktioner och svar på frågor noterades.

\section{Möten med fysikläraren}

För att få återkoppling på läromaterialet hölls två möten med Åke Fäldt. Ett möte hölls relativt
tidigt i projektet, 2018-03-02, och ett andra relativt sent, 2018-04-11.
Under mötena presenterades läromaterialet i sig och tanken med det, nämligen att
presentera fysik ur ett annat perspektiv, ett
programmeringsperspektiv. Diskussionen kretsade sen kring svåra områden i Fysik för
ingenjörer (se avsnitt \ref{sec:kontakt_faldt}), vad läromaterialet skulle kunna
bidra med för kunskaper till studenter samt dess eventuella roll i relation till
fysikkursen.
