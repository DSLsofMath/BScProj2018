
\chapter{Resultat}

\begin{binge}

\section{Läromaterialet}

Läromaterialet blev i slutändan en sammanvävning av domänspecifika språk som modellerar fysik, och lärotext som förklarar kopplingen mellan fysiken och de domänspecifika språken. Figur \ref{fig:smakprov_laromaterial} visar ett kort utdrag ur läromaterialet. Där ser man hur domänspecifika språk och lärotext är sammanvävda.

\begin{figure}[tph]
  \includegraphics[width=\linewidth]{figure/smakprov_laromaterial.png}
  \caption{Ett smakprov över hur det resulterande läromaterialet ser ut. Lärotext är framför den ljusgrå bakgrunden medan Haskell-kod för domänspecifika språk är framför den mörkgrå.}
  \label{fig:smakprov_laromaterial}
\end{figure}

Läromaterialet innehåller även bilder. Figur \ref{fig:smakprov_bild_laromaterial} är ett exempel på en bild ur läromaterialet. Speciellt att notera är den medvetet oseriösa ritningstekniken som är tänkt att vara rolig och muntra upp läsaren.

\begin{figure}
  \includegraphics[width=\linewidth]{figure/smakprov_bild_laromaterial.png}
  \caption{Exempel på en bild ur läromaterialet. Bilden visar hur en hund springer och hoppar upp på en stillastående vagn. När hunden landat rör sig hunden och vagnen med en ny, gemensam hastighet.}
  \label{fig:smakprov_bild_laromaterial}
\end{figure}

Läromataterialet behandlar ett flertal områden inom fysik och matematik som används inom fysik. Fokuset är på klassisk mekanik samt till det området tillhörande matematik. I sin fullständighet är de behandlade områdena

\begin{itemize}
  \item Analys
  \item Bevis
  \item Dimensioner
  \item Fysikaliska kroppar
  \item Vektorer
\end{itemize}

\textit{Analys} handlar om matematisk analys och bygger upp ett syntaxträd... TODO: kolla mer detaljrikt.

I \textit{Bevis}-kapitlet presenteras bevisföringen med hjälp av Haskells typsystem. Det exemplifieras genom att kinematiska formler bevisas.

\textit{Dimensioner} behandlar dimensioner, storheter och enheter inom fysiken. Dimensioner införs på typnivå i Haskell för att kunna visa på likheten mellan Haskells typsystem och hur man måste förhålla sig till dimensioner inom fysiken.

\textit{Fysikaliska kroppar} vet ej TODO: ta reda på

\textit{Vektorer} vet ej TODO: ta reda på

I läromataterialet finns, förutom de fem ovanstående separata områdena, även tillämpningar av de områdena på exempelproblem. Till exempel används \textit{Dimensioner} till att lösa problem med fritt fall. TODO: fyll på när vet mer om hur tillämpningarna ser ut.

\subsubsection{Domänspecifika språk}
TODO: Flytta (ta bort) till relevanta områden (Teori/Metod)

De domänspecifika språken modellerar områden snarare än att vara problemlösare.
\textit{Analys} exemplifierar detta väl. Det språket består av ett syntaxträd
över algebraiska uttryck samt operationer som derivering och integration. Med
hjälp av det kan man modellera uttryck och analytiska operationer på dem.
Däremot löser det inte problem åt en. Man kan med andra ord inte mata in en
differentialekvation och automatiskt få en lösning.

TODO: Detta kanske passar bättre i diskussion: Hur bra våra moduler blev samt varför de inte är problemlösare utan modellerare istället.

\subsubsection{Lärotexten}

Texten är skriven på engelska för att komma fler till gagn än om den varit
skriven på svenska.

Språket i texten är vardagligt och lättsamt. Detta för att vara en kontrast mot
hur kursböcker vanligtvis ser ut.

I brödtexten finns bilder. De har en rolig och medvetet kladdig stil. Syftet är
att muntra upp läsaren. Figur \ref{fig:smakprov_bild_laromaterial} är ett
exempel på en bild som finns i läromaterialet. Den visar den humoristiska och
oseriösa ritningstekniken.



\subsection{Hemsidan}

Läromaterialet finns tillgängligt på en hemsida, som består av grundläggande
HTML, CSS och javascript. På hemsidan finns en innehållsförteckning med
klickbara länkar till de olika kapitlen. Hemsidan är öppen för alla och bör
fungera i de flesta webläsare. Javascript är inget krav för hemsidan.
Matematiska formler visas ändå, om än inte lika tydligt.

\subsection{Källkod}

All källkod finns fritt tillgänglig att läsa och kopiera via internet.
\ref{lyap}

\section{Återkoppling från testgrupp}

För att utvärdera huruvida läromaterialet är intressant och hjälpsamt har vi
haft en informell återkoppling med en testgrupp. TODO: Skriv mer när vi
faktiskt gjort detta.

\end{binge}
