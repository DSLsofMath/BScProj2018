
\chapter{Resultat om domänspecifika språk och fysik}

\begin{draft}

I detta kapitel redovisas resultaten från möten med testgrupp och Åke Fäldt. De diskuteras och tolkas utförligare i nästa kapitel, kapitel \ref{cha:disk2}.

\section{Utvärderingen med testgruppen}

TODO: Anapassa till det nya syftet.

Utfallet från utvärderingen med testgruppen var till övervägande del positivt. Testgruppen tyckte läromaterialet var ett intressant och roligt sätt att presentera fysik på. De tyckte att bilderna tjänade sitt syfte, att muntra upp läsaren. 

En poäng som framfördes var att inte börja kapitlena för komplicerat. Istället tyckte de att det skulle vara bra att börja enkelt, för att kunna hänga med i både Haskell-koden och fysiken, för att därefter behandla ett område mer detaljerat. Att visa ett kort exempel i Haskell för att sedan låta läsaren själv göra något liknande var ett förslag på hur det kunde göras.

Utvärderingen var dock för kort för att det skulle framgå huruvida läsaren lärde sig mest fysik eller mest Haskell. Det framgick heller inte om läromatarialet uppmuntrade testgruppen att vilja lära sig mer fysik.

\section{Möten med Fäldt}

TODO: Vad som framgick från mötena.

\end{draft}






























