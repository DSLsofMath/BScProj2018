% CREATED BY DAVID FRISK, 2018

% IMPORT SETTINGS
\documentclass[12pt,a4paper]{article}
\usepackage[utf8]{inputenc}

\setlength{\parindent}{0pt}
\setlength{\parskip}{\baselineskip}
\usepackage[a4paper, margin=1in]{geometry}

\usepackage[noconfigs, swedish]{babel}

\begin{document} 

\pagenumbering{gobble}

\title{Titel}
\date{\today}
\author{Björn Werner\\ Erik Sjöström \\ Johan Johansson \\ Oskar Lundström}

\maketitle

% TABLE OF CONTENTS
\newpage
\tableofcontents

% START OF MAIN DOCUMENT
\newpage
\setcounter{page}{1}
\pagenumbering{arabic}			% Arabic numbering starting from 1 (one)
\setlength{\parskip}{0pt plus 1pt}

\section{Inledning}

Bakgrund I detta avsnitt beskrivs bakgrunden till frågeställningen,
det vill säga en kort beskrivning av företagets situation och varför man vill
ha uppdraget utfört. Observera att det inte är bakgrunden till att ni gör arbetet
som ska beskrivas här.
DSLs är nyttiga för att studenter ska förstå saker ur olika perspektiv.

\section{Bakgrund}

\begin{itemize}
    \item Vad?
    \item Varför?
    \item Relevant för vem?
    \item Kan det relateras till något område?
\end{itemize}

\section{Syfte}

Syftet är en kort beskrivning av uppdraget och vilket resultat som uppdraget
ska leda till.
Projektets syfte är att implementera ett antal DSLs i ett par olika discipliner, samt
en dokumentation över användningen av dom.

\begin{itemize}
    \item 1-2 meningar
    \item Vad ska resultatet vara?
\end{itemize}

\section{Problem/Uppgift}

Precisering av frågeställningen Utifrån "Syfte" ska frågeställningen preciseras.
Detta kan göras t ex genom att ställa upp ett antal frågor som ska besvaras. Ett
annat sätt är att ange ett antal påstående (hypoteser) som sedan ska verifieras eller
förkastas under arbetets gång.

\begin{itemize}
    \item Problemanalys
    \item Bryt ner i mindre delar
\end{itemize}

\section{Avgränsningar}

Under avgränsningar talar man om vad man inte behandlar.
Projekter avgränsar sig till exempelvis att implementera DSLs för samtliga områden
inom kursen Fysik för ingengörer.

\begin{itemize}
    \item Vilka delar av det övergripande syftet som ej ska med
\end{itemize}

\section{Metod/Genomförande}

Metodkapitlet ska beskriva hur man avser att lägga upp arbetet. Detta
omfattar bland annat arbetsgång, design av experiment och användning av olika
datainsamlingsmetoder. Ett metodkapitel ska, i idealfallet, vara så utförligt att
vem som helst som har vissa baskunskaper inom området ska kunna utföra arbetet
på det sättet som har beskrivits i rapporten och nå samma resultat. Att beskriva
metoden är viktigt för att uppdragsgivaren ska kunna bedöma om man kan nå målet
på det föreskrivna sättet. Det är därför också viktigt att man förklarar varför den
valda metoden ger ett tillförlitligt resultat.

\section{Samhälleliga och etiska aspekter}
%\section{Hållbar utveckling}
% Byte namn så att bilagan från kandidatriktlinjerna följs istället

\section{Tidsplan}

Tidplanen presenteras lämpligen i form av ett Gantt-schema.

\begin{itemize}
    \item Vad är när saker ska göras
    \item Var konkret och detaljerad
\end{itemize}

\end{document}