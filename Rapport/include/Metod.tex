% Metoddelen redogör för vad du gjort och hur du gått tillväga; det är
% en beskrivning av den metod som ligger till grund för det du kommit
% fram till och hävdar i din rapport.

% Beskrivningen i metoddelen ska vara koncis snarare än helt
% uttömmande men ska samtidigt göra det möjligt att upprepa studien

% Metoddelen ska inte vara en omgjord labbinstruktion och den ska inte
% heller innehålla teori med mindre än att teoretiska hänsyn har haft
% en direkt inverkan på metoden.

% Metoddelen skrivs nästan alltid i dåtid (imperfekt) och ofta används
% passiv form för att beskriva forskningsaktiviteter.

\chapter{Metod/Genomförande}

% TODO: Är det intressant att ha med HUR vi, genom research, kom fram
% till att göra en specifik sak och varför den är bra?
%
% E.g. "Genom att läsa i bok A om område B så kom vi fram till att vi
% ska applicera metod C för att lösa problem D. Enligt bok A är metod
% C bra därför att ..."
% vs.
% "Vi valde att lösa problem D genom att applicera metod C. Metod C är
% bra därför att ...[ev. källhänvisning]."

\begin{binge}
Skapandet av läromaterialet har i grova drag haft tre faser. Först valdes olika
arbetsområden ut, som enskilt gick att arbeta med. Sedan Skapades läromaterial
för dessa områden. Till sist sammanfoagdes resultatet.

% TODO: Varför på detta viset? Lite mer om strukturen både i hur vi
% arbetade, och hur läromaterialet ser ut.
% E.g. "1. identifiera problemområden; 2. impl dsl; 3. Skriv lärotext
% om DSL; 4. Visa hur DSL kan appliceras för att lösa problem"

% TODO: Sektion om våra litteraturstudier? Känns lite skumt att ha,
% men 2016 hade. Litteratur är väl bara intressant att ha med som
% refens när man refererar till fakta eller motiverar ett val? Själva
% studieprocessen, i.e. HUR man nådde beslutet, är väl inte alls lika
% intressant att ha med som VARFÖR man tog beslutet?

\section{Selektion av arbetsområden}

% TODO: Åke mailet hjälpte oss identifiera områden.
% TODO: Hjälpte det senare Åke mötet nåt? För just detta?

För att hitta områden att arbeta med studerades främst kursboken *Univeristy
Physics*. Där lästes de kapitel som ingick i *Fysik för ingenjörer*. Sådant som
verkade hade syntax som behövde förklaras, eller sådant som var svårt, eller
sådant som var spännande valdes ut. De områden som hittats sorterades upp i
grupper som var så fristående som möjligt för att kunna arbetas med på
parallellt.

De områden som valdes ut blev

- Vektorer

- Enheter

- Momentan och genomsnitt

- Differentalkalkyl

% TODO: Nåt om hur områdesvalen speglas i strukturen av läromaterialet?

\section{Implementation av DSLer för områdena}

\subsection{Skriva DSL för ett område}

Varje gruppmedlem valde varsitt område att arbeta med. Man började med att
experimentera med DSL:et för att hitta bra sätt att representera området på,
vad som var tydligt och lätthanterat i datorn.

Det skedde också en del Haskell-inläsning av nya områden, exempelvis
typnivå-progammering, för att kunna göra DSL:er på bästa sätt.

När en tanke börjat formas så implementerades först DSL:et. När det till stora
delar var klart började förklarande brödtext skrivas till det, främst för att
förklara koden som skrivits.

När koden var färdig och kommenterad tillräckligt väl började brödtexten
uppdaters för att även innehålla mer kopplingar till fysik.

\subsection{Komposita områden}

Importera DSLerna för varandra för att göra mer komplicerade grejer.

Eller, \emph{Bruk av de mer fundamentala/teoretiska DSLerna för att
  angripa områden av mer ``tillämpad'' natur (såsom Krafter, Arbete,
  etc) )}(?)

!! Områden/moduler soom bygger vidare på redan implementerade områden.

stack, git kan säkert passa här för att beskriva hur vi samarbetade med sammanfogningen.

\section{Skriva lärotext till DSLer}

\subsection{Didaktik/språk/utlärningsmetod}

Lättsamt språk o en gnutta humor för att hålla kvar
uppmärksamhet. Relaterat till Attention i ARCS modellen (2016 skrev om
det, såg vettigt ut).

\section{Skriva läromaterial för hur DSLer appliceras för problemlösning}

Vari vi visar att DSLerna både är praktiskt användbara, likt Wolfram
Alpha, och att implementationen+applikationen hjälper oss förstå
mekanik i allmänhet och probleminstanserna i synnerhet.

% TODO: Motivera varför vi valde LHS
% TODO: Motivera varför vi valde hemsida istället för PDF.
% TODO: Motivera varför vi valde Markdown istället för LaTeX i LHS filerna.

\section{Publicering/presentation}

Läromaterialet publiceras på en internethemsida, varpå man kan läsa
allt o ha skoj.

\subsection{Beskrivning}

Ett build-script hämtar .lhs källfilerna, i vilka brödtexten är
skriven med markdown. Rendrar med pandoc, och sätter in lite
navigationselement etc. med hjälp av eget templating-system. Manuellt
läggs sedan stoffet på gh-pages branchen för att automatiskt visas på
dslsofmath.github.io/BScProj2018.

Obs: Medan bygget är scriptat så är inte publiceringen det, och
ingenting genereras/publiceras automatiskt kontinuerligt. Måste köra
scriptet manuellt och lägga stoff på gh-pages branchen.

\subsection{Build-script}

I.e. implementation av python-build-scriptet i mer detalj.

Är detta ens intressant? Viktigt för att producera sidan såklart, men
inte intressant ur varken matte eller haskell/DSL perspektiv.

\subsection{Hemsidan}

Nåt om design, läslighet, grafik(?), navigation, avsiktligt undvikande
av javascript, etc.

Tänker lite samma med denna sektion som ovan. Har ju ingenting med
varken matte eller DSL att göra i sig, så kanske inte så intressant?
Samtidigt är det kanske lite intressant ur pedagogik-aspekten. Kan det
kanske vara lättare/roligare att lära sig om sidan är fin och
lättläst? Att javascript inte krävs gör att sidan kan visas ordentligt
även om man sitter i U-land med dålig/gammal/billig telefon.

\section{Test och återkoppling}

Test på försöksstudenter. Återkoppling med Åke(?).

Nog bra att vara explicit här med att det inte är en rigorös empirisk
studio, om inte det redan täckts väl i Avgränsningar.

\end{binge}
