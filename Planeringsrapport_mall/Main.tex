% CREATED BY DAVID FRISK, 2017

% IMPORT SETTINGS
\documentclass[12pt,a4paper,twoside,openright]{report}
\input{include/settings/Settings}

\begin{document} 
\textbf{Planeringsrapporten}

Planeringsrapporten ska beskriva vad som ska göras och hur det ska genomföras. Planeringsrapporten ska omfatta en precisering av syftet, en beskrivning av hur arbetet ska utföras samt tidplanen för arbetets genomförande, det vill säga svara på frågorna vad? hur? när?.
Observera att planeringsrapporten ska vara godkänd av företaget och av handledaren på skolan innan det egentliga arbetet får starta. Planeringsrapporten ska omfatta följande kapitel:

1. INLEDNING
1.1 Bakgrund
I detta avsnitt beskrivs bakgrunden till frågeställningen, det vill säga en kort beskrivning av företagets situation och varför man vill ha uppdraget utfört. Observera att det inte är bakgrunden till att ni gör arbetet som ska beskrivas här.

DSLs är nyttiga för att studenter ska förstå saker ur olika perspektiv.

1.2 Syfte
Syftet är en kort beskrivning av uppdraget och vilket resultat som uppdraget ska leda till.

Projektets syfte är att implementera ett antal DSLs i ett par olika discipliner, samt en dokumentation över användningen av dom.

1.3 Avgränsningar
Under avgränsningar talar man om vad man inte behandlar.

Projekter avgränsar sig till exempelvis att implementera DSLs för samtliga områden inom kursen Fysik för ingengörer.

1.4 Precisering av frågeställningen
Utifrån "Syfte" ska frågeställningen preciseras. Detta kan göras t ex genom att ställa upp ett antal frågor som ska besvaras. Ett annat sätt är att ange ett antal påstående (hypoteser) som sedan ska verifieras eller förkastas under arbetets gång.

2. METOD
Metodkapitlet ska beskriva hur man avser att lägga upp arbetet. Detta omfattar bland annat arbetsgång, design av experiment och användning av olika datainsamlingsmetoder. Ett metodkapitel ska, i idealfallet, vara så utförligt att vem som helst som har vissa baskunskaper inom området ska kunna utföra arbetet på det sättet som har beskrivits i rapporten och nå samma resultat. Att beskriva metoden är viktigt för att uppdragsgivaren ska kunna bedöma om man kan nå målet på det föreskrivna sättet. Det är därför också viktigt att man förklarar varför den valda metoden ger ett tillförlitligt resultat.

3. TIDPLAN
Tidplanen presenteras lämpligen i form av ett Gantt-schema.



\end{document}
