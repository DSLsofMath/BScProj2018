\documentclass[DIV16,twocolumn,10pt]{scrreprt}
\usepackage{paralist}
\usepackage{graphicx}
\usepackage[utf8]{inputenc}
\usepackage[final]{hcar}

%include polycode.fmt

\begin{document}

\begin{hcarentry}{Learn You A Physics}
\report{Erik Sjöström}
\status{active development}
\participants{Oskar Lundström, Johan Johansson, Björn Werner}% optional
\makeheader

Learn You A Physics is the result of a BSc project at Chalmers (supervised by P. Jansson) where the
goal is to create an introductory learning material for physics aimed at programmers
with a basic understanding of Haskell.

It does this by identifying key areas in physics with a well defined scope,
for example dimensional analysis or single particle mechanics, and develops
a domain specific language around each of these areas.

The implementation of these DSL's are the meat of the learning material with
accompanying text to explain every step and how it relates to the physics of
that specific area.

The text is written in such a way as to be as non-frightening as possible,
and to only require a beginner knowledge in Haskell.
%
Inspiration is taken from \href{http://learnyouahaskell.com/}{Learn you a Haskell for Great Good} and the project \href{https://github.com/DSLsofMath/DSLsofMath}{DSLsofMath} at Chalmers and University of Gothenburg.

The \href{https://github.com/DSLsofMath/BScProj2018/tree/master/Physics}{source
code} and \href{https://dslsofmath.github.io/BScProj2018/}{learning material}
is freely available online.

\FurtherReading
  \href{https://dslsofmath.github.io/BScProj2018/} {Learn You A Physics}
\end{hcarentry}

\end{document}
