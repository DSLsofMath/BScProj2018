
\chapter{Metod2/Genomförande2: Beskrivning av konstruktionen av läromaterialet}
\begin{binge}

Arbetet med att skapa läromaterialet har haft ett antal delvis parallella faser. Den första fasen var att skapa läromaterial till separata områden inom fysik. Den andra fasen var att kombinera de tidigare DSLerna till komposita områden. Den tredje fasen var att använda tidigare DSLer och tillämpa på fysikaliska problem. Den fjärde fasen var att publicera läromaterial på en hemsida.

Den första fasen var tvunget att ske innan någon annan, då de andra beror på den första. Men efter att lite olika fristående DSLer skapats kunde de tre andra faserna börja ske parallellt. Men den första fasen fortsattae. Olika medlemmar i gruppen arbetade med olika faser samtidigt.

\section{Läromaterial för separata områden}

Processen med att skapa läromaterial för separata områden har varit en loopande process.

\begin{figure}
  \includegraphics[width=\linewidth]{figure/floedesplan.jpg}
  \caption{Översikt över processen med att skapa de separata områdena.}
  \label{fig:oeversikt_separata}
\end{figure}

\subsection{Selektion av områden}

Att hitta områden att arbeta med inom fysik var inget trivialt att göra. De främsta källorna till inspiration var kursboken och annat material som tillhörde kursen. Ett område skulle vara fristående från andra områden, vara grundläggande samt vara ''lämpligt''.

Vad för slags områden detta var, var inget som gick att se bara genom att titta på det. Istället fick det experimenteras med området för att upptäcka om det var ''lämpligt'', dvs gick att representera på ett lätthanterligt och bra sätt i datorn. Dessa krav specifiefas närmare i resultat-kapitelt (då vi visste vad dessa karaktäristika var).

Sökandet efter ett område har utförsts flera gånger i projeket. Ofta skedde det individuellt då. Ett speciellt fall av sökande är dock den initiala selektionen, som skedde i början av projektfasen. Då lästes hela kursboken och allt innehåll radades upp och sorterades. På detta sätt kunde några initiala områden väljas ut att arbteras med. Dessa områden blev

\begin{itemize}
  \item Vektorer
  \item Dimensioner
  \item Momentan och genomsnitt
  \item Differentialkalkyl
\end{itemize}

I samband med experiment med DSL skedde också en del inläsning, av bland annat Haskell och Agda. Ett exempel var typnivå-programmering. Detta gjordes för att kunna göra DSL:et på bästa sätt.

Långt ifrån alla försök med områden blev lyckade. Märktes att ett område inte skulle bli något bra DSL så började vi om processen med att hitta ett nytt område. Som tidigare nämnt, så står det vad som var ett bra område och vad som var ett dåligt senare i resultat-kapitlet.

\subsection{''renskrivning'' av lämpliga områden}

\section{Läromaterial för komposita områden}

\todo{Här har vi inte gjort så mycket än eller?}

\section{Tillämpning av DSL:er för problemområden}

Fritt fall, lutande plan...

\section{Publicering på hemsidan}

Först en inledande konstruktion av hemsidan.

Sedan löpande att material kompileras och läggs där.

\end{binge}
