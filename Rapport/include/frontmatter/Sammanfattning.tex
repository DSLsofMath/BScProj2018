
\thispagestyle{plain}			% Supress header

TODO: Perhaps include both Abstract + Sammanfattning on one page?

\begin{binge}

\section*{Sammandrag}

% SYFTE
Detta projekt är ett kandidatarbete vid instutuionen för Data- och
informationsteknik på Chalmers tekniska högskola. Målet med projektet är att
konstruera ett läromaterial i textform som presenterar fysik med hjälp av
domänspecifika språk i Haskell. I denna rapport beskrivs utvecklingen av
läromaterialet, hur det ser ut samt ett par aspekter kring kombinationen av
domänspecifika språk och fysik.

% BAKGRUND
Bakgrunden till projektet är den obligatoriska fysikkursen \textit{Fysik för
ingenjörer} som ges i årskurs 2 på Datateknik på Chalmers. Kursen har enligt oss
en mindre bra tentastatistik och en oklar koppling till det resterande
programmet. Dessa två problem är tänkta att lösas med ett läromaterial som
fungerar som en brygga mellan fysik och programmering, och visar då både
relevansen med och väcker intresse för fysik. Ett ökat intresse för fysik leder
förhoppningsvis till bättre resultat i kursen.

% RESULTAT
Det resulterande läromaterialet innehåller X kapitel som behandlar områdena
bevis, fysikalsika dimensioner, matematisk analys, vektorer, partikelmekanik och
tillämpningar av dem. Vissa kapitel bygger upp domänspecifika språk från grunden
medan andra kombinerar och tillämpar tidigare domänspecifika språk på
fysikaliska problem. Läromaterialet publicerades på en hemsida:
\url{https://dslsofmath.github.io/BScProj2018/} och dess källkod finns fritt
tillgänglig: \url{https://github.com/DSLsofMath/BScProj2018}.

% DISKUSSION
Domänspecifika språk kan ha en pedagogisk nytta i fysikundervisning. Eftersom de
domänspecifika språken är rigorösa leder till att den fysikalsika
problemlösningen i dem också blir rigorös, utan möjlighet till genvägar. Detta
tankesätt är lärorikt att förmedla till klassisk fysikalsika undervisning.

% KEYWORDS (MAXIMUM 10 WORDS)
\vfill
Nyckelord: Domain specific language, Classical mechanics, Teaching material, Functional programming, Classical physics

\end{binge}

\newpage				% Create empty back of side
\thispagestyle{empty}
\mbox{}
